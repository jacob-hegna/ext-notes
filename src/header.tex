%% LaTeX header file for Math 821, Algebraic Topology
%% JLM 1/9/18
%% Feel free to use or modify these macros as you see fit.

% General stuff
\documentclass[reqno]{amsart}
\usepackage{amssymb,amsmath,amsthm,graphics,hyperref,mathrsfs,stmaryrd,dsfont,youngtab,tcolorbox,framed,blkarray,color,fullpage}
\usepackage[all]{xy}
\usepackage{enumerate}
\usepackage{bbm}
\numberwithin{equation}{section}
\renewcommand{\phi}{\varphi}

%%%%%%%%%%%%%%%%    Macros for figures   %%%%%%%%%%%%%%%%

%% I like to keep all my figures in a subfolder called "figs".  If you keep figures in the same folder as the source .tex tile, remove "figs/" from the macros.

%% Usage: \includefigure{height}{width}{filename}
\newcommand{\includefigure}[3]{{
  \begin{center}
  \resizebox{#1}{#2}{\includegraphics{{figs/#3}}}
  \end{center}}}
\newcommand{\includefigurewithinmath}[3]{{
  \resizebox{#1}{#2}{\includegraphics{{figs/#3}}}}}

%%%%%%%%%%%%%%%%   Vertical spacing   %%%%%%%%%%%%%%%%
\newcommand{\hugepad}{\rule[-14mm]{0mm}{30mm}}
\newcommand{\bigpad}{\rule[-5mm]{0mm}{15mm}}
\newcommand{\bigpadup}{\rule{0mm}{10mm}}
\newcommand{\bigpaddown}{\rule[-5mm]{0mm}{5mm}}
\newcommand{\smallpad}{\rule[-1.5mm]{0mm}{5mm}}
\newcommand{\pad}{\rule[-3mm]{0mm}{8mm}}
\newcommand{\padup}{\rule{0mm}{5mm}}
\newcommand{\paddown}{\rule[-3mm]{0mm}{2mm}}
\newcommand{\blank}{\rule{1.25in}{0.25mm}}
\newcommand{\yell}[1]{\begin{framed}{#1}\end{framed}}
\newcommand{\bang}{$\bullet$\quad}
\newcommand{\indnt}{\phantom{.}\qquad}
\newcommand{\littleline}{\begin{center}\rule{4in}{0.5bp}\end{center}}
\newcommand{\defterm}[1]{\boldmath\textbf{#1}\unboldmath}

\newcommand{\icol}[1]{% inline column vector
  \left(\begin{smallmatrix}#1\end{smallmatrix}\right)%
}

%%%%%%%%%%%%%%%%   Colors for TikZ and text  %%%%%%%%%%%%%%%%%

\definecolor{light}{gray}{.75}
\definecolor{med}{gray}{.5}
\definecolor{dark}{gray}{.25}
\newcommand{\Red}[1]{{\color{red}{#1}}}
\newcommand{\RED}[1]{{\color{red}{\boldmath\textbf{#1}\unboldmath}}}
\newcommand{\Blue}[1]{{\color{blue}{#1}}}
\newcommand{\BLUE}[1]{{\color{blue}{\boldmath\textbf{#1}\unboldmath}}}

%%%%%%%%%%%%%%%%%%%%   Hyperlinks   %%%%%%%%%%%%%%%%%%%%

\newcommand{\hreftext}[2]{\href{#1}{\Blue{#2}}}
\newcommand{\hrefurl}[2]{\href{#1}{\Blue{\tt #2}}}

%%%%%%%%%%%%%%%%%%%%   Math operators   %%%%%%%%%%%%%%%%%%%%

\DeclareMathOperator{\Spec}{Spec}
\DeclareMathOperator{\interior}{Int}
\DeclareMathOperator{\Mod}{Mod}
\newcommand{\etale}{\'etal{e\xspace}}

\DeclareMathOperator{\Ab}{Ab}
\DeclareMathOperator{\coker}{coker}
\DeclareMathOperator{\Comm}{Comm}
\DeclareMathOperator{\conv}{conv}
\DeclareMathOperator{\diag}{diag}
\DeclareMathOperator{\Ext}{Ext}
\DeclareMathOperator{\Gr}{Gr}
\DeclareMathOperator{\Hom}{Hom}
\DeclareMathOperator{\im}{im}
\DeclareMathOperator{\Link}{link}
\DeclareMathOperator{\nullspace}{nullspace}
\DeclareMathOperator{\Poin}{Poin}
\DeclareMathOperator{\rank}{rank}
\DeclareMathOperator{\rel}{rel}
\DeclareMathOperator{\Tor}{Tor}
\DeclareMathOperator{\tr}{tr}
\DeclareMathOperator{\sd}{sd}
\DeclareMathOperator{\Star}{star}
\DeclareMathOperator{\support}{support}
\DeclareMathOperator{\vol}{vol}

%%%%%%%%%%%%%%%%%%%%   Theorem environments with automatic numbering

\newtheorem{theorem}{Theorem}[section]
\newtheorem{proposition}[theorem]{Proposition}
\newtheorem{lemma}[theorem]{Lemma}
\newtheorem{corollary}[theorem]{Corollary}
\newtheorem{criterion}[theorem]{Criterion}
\theoremstyle{definition}
\newtheorem{definition}[theorem]{Definition}
\newtheorem{example}[theorem]{Example}
\newtheorem{remark}[theorem]{Remark}
\newtheorem{problem}[theorem]{Problem}
\theoremstyle{remark}
\newtheorem{exercise}{Exercise}

%%%%%%%%%%%%%%%%%%%%   Unnumbered theorem proclamations

\newcommand{\cor}{{\bf Corollary: }}
\newcommand{\defn}{{\bf Definition: }}
\newcommand{\defnthm}{{\bf Definition/Theorem: }}
\newcommand{\defnprop}{{\bf Definition/Proposition: }}
\newcommand{\defns}{{\bf Definitions: }}
\newcommand{\exa}{{\bf Example: }}
\newcommand{\fact}{{\bf Fact: }}
\newcommand{\lem}{{\bf Lemma: }}
\newcommand{\notn}{{\bf Notation: }}
\newcommand{\obs}{{\bf Observation: }}
\newcommand{\note}{{\bf Note: }}
\newcommand{\prop}{{\bf Proposition: }}
\newcommand{\rmk}{{\bf Remark: }}
\newcommand{\skpr}{\emph{Sketch of proof: }}
\newcommand{\soln}{{\bf Solution: }}
\newcommand{\altsoln}{{\bf Alternate Solution: }}
\newcommand{\altaltsoln}{{\bf Another Alternate Solution: }}
\newcommand{\thm}{{\bf Theorem: }}

\newcommand{\half}{\frac{1}{2}}
\newcommand{\thalf}{\tfrac{1}{2}}
\newcommand{\dhalf}{\dfrac{1}{2}}

\newcommand{\0}{\emptyset}
\newcommand{\bd}{\partial} %% boundary
\newcommand{\dju}{\ensuremath{\mathaccent\cdot\cup}} %% disjoint union
\newcommand{\eqdef}{\overset{\rm def}{=}}
\newcommand{\ergo}{\therefore}  %% just because it's shorter
\newcommand{\excise}[1]{} % for commenting out large blocks of LaTeX source
\newcommand{\intersection}{\cap}
\newcommand{\isom}{\cong}  %% homeomorphism
\newcommand{\htop}{\simeq} %% homotopy equivalence
\newcommand{\ov}[1]{\overline{{#1}}}
\newcommand{\qandq}{\quad\text{and}\quad}
\newcommand{\qqandqq}{\qquad\text{and}\qquad}
\newcommand{\qand}{\quad\text{and}}
\newcommand{\ray}{\overrightarrow}
\newcommand{\rchi}{\raisebox{0.05cm}{\ensuremath{\chi}}}  %% because chi looks funny otherwise
\newcommand{\sm}{\setminus}
\newcommand{\st}{~|~}		%% ``such that''
\newcommand{\www}{\mathsf w}	%% winding number
\newcommand{\x}{\times}
\newcommand{\union}{\cup}
\newcommand{\zerovec}{\mathbf{0}}

%% Arrows
\newcommand{\into}{\hookrightarrow} % injection
\newcommand{\onto}{\twoheadrightarrow} % surjection
\newcommand{\inj}{\hookrightarrow} % injection
\newcommand{\surj}{\twoheadrightarrow} % surjection

%% Arrows for use with xy
\newcommand{\rto}{\ar[r]}
\newcommand{\rrto}{\ar[rr]}
\newcommand{\lto}{\ar[l]}
\newcommand{\llto}{\ar[ll]}
\newcommand{\uto}{\ar[u]}
\newcommand{\uuto}{\ar[uu]}
\newcommand{\dto}{\ar[d]}
\newcommand{\ddto}{\ar[dd]}
\newcommand{\dlto}{\ar[dl]}
\newcommand{\drto}{\ar[dr]}
\newcommand{\ulto}{\ar[ul]}
\newcommand{\urto}{\ar[ur]}
\newcommand{\ddlto}{\ar[ddl]}
\newcommand{\dllto}{\ar[dll]}
\newcommand{\ddrto}{\ar[ddr]}
\newcommand{\drrto}{\ar[drr]}
\newcommand{\uulto}{\ar[uul]}
\newcommand{\ullto}{\ar[ull]}
\newcommand{\uurto}{\ar[uur]}
\newcommand{\urrto}{\ar[urr]}
\newcommand{\ddllto}{\ar[ddll]}
\newcommand{\uullto}{\ar[uull]}

%% Straight lines for use with xy
\newcommand{\rlineto}{\ar@{-}[r]}
\newcommand{\rrlineto}{\ar@{-}[rr]}
\newcommand{\llineto}{\ar@{-}[l]}
\newcommand{\lllineto}{\ar@{-}[ll]}
\newcommand{\ulineto}{\ar@{-}[u]}
\newcommand{\uulineto}{\ar@{-}[uu]}
\newcommand{\dlineto}{\ar@{-}[d]}
\newcommand{\ddlineto}{\ar@{-}[dd]}
\newcommand{\dllineto}{\ar@{-}[dl]}
\newcommand{\drlineto}{\ar@{-}[dr]}
\newcommand{\ullineto}{\ar@{-}[ul]}
\newcommand{\urlineto}{\ar@{-}[ur]}
\newcommand{\ddllineto}{\ar@{-}[ddl]}
\newcommand{\dlllineto}{\ar@{-}[dll]}
\newcommand{\ddrlineto}{\ar@{-}[ddr]}
\newcommand{\drrlineto}{\ar@{-}[drr]}
\newcommand{\uullineto}{\ar@{-}[uul]}
\newcommand{\ulllineto}{\ar@{-}[ull]}
\newcommand{\uurlineto}{\ar@{-}[uur]}
\newcommand{\urrlineto}{\ar@{-}[urr]}

%% Boldface letters
\renewcommand{\aa}{\mathbf{a}}
\newcommand{\bb}{\mathbf{b}}
\newcommand{\pp}{\mathbf{p}}
\newcommand{\qq}{\mathbf{q}}
\newcommand{\vv}{\mathbf{v}}
\newcommand{\ww}{\mathbf{w}}
\newcommand{\xx}{\mathbf{x}}
\newcommand{\yy}{\mathbf{y}}
\newcommand{\zz}{\mathbf{z}}

%% Calligraphic letters
\newcommand{\A}{\mathcal{A}}
\newcommand{\B}{\mathcal{B}}
\newcommand{\C}{\mathcal{C}}
\newcommand{\M}{\mathcal{M}}
\renewcommand{\P}{\mathcal{P}}
\newcommand{\T}{\mathcal{T}}

%% Fancy script letters
\newcommand{\BB}{\mathscr{B}}
\newcommand{\CC}{\mathscr{C}}
\newcommand{\FF}{\mathscr{F}}
\newcommand{\II}{\mathscr{I}}
\newcommand{\LL}{\mathscr{L}}
\newcommand{\PP}{\mathscr{P}}
\renewcommand{\SS}{\mathscr{S}}
\newcommand{\TT}{\mathscr{T}}
\newcommand{\XX}{\mathscr{X}}

%% Blackboard bold letters
\newcommand{\Aa}{\mathbb{A}}
\newcommand{\Cc}{\mathbb{C}}
\newcommand{\Ff}{\mathbb{F}}
\newcommand{\Nn}{\mathbb{N}}
\newcommand{\Pp}{\mathbb{P}}
\newcommand{\Qq}{\mathbb{Q}}
\newcommand{\Rr}{\mathbb{R}}
\newcommand{\Ss}{\mathbb{S}}
\newcommand{\Zz}{\mathbb{Z}}
\newcommand{\IdMap}{\mathds{1}}

\newcommand{\RP}{\Rr P}
\newcommand{\CP}{\Cc P}
\newcommand{\RPN}{\Rr P^n}
\newcommand{\CPN}{\Cc P^n}
\newcommand{\RPn}{\Rr P^n}
\newcommand{\CPn}{\Cc P^n}

%% tilde H for reduced homology
\newcommand{\HH}{\tilde H}

%% big double angle brackets
\newcommand{\Langle}{\big\langle\!\!\big\langle}
\newcommand{\Rangle}{\big\rangle\!\!\big\rangle}

%% macros for automatic problem numbering
\newcounter{probno}
\setcounter{probno}{0}
\newcounter{partno}
\setcounter{partno}{0}

%% versions that don't print the number of points
\newcommand{\yellprob}[1]{%
  \bigskip\begin{framed}
  \setcounter{partno}{0}%
  \addtocounter{probno}{1}%
  {\bf Problem~\#{\arabic{probno}}}\quad\bf\boldmath #1\unboldmath
  \end{framed}}

\newcommand{\yellprobpart}[1]{
  \bigskip\begin{framed}%
  \addtocounter{partno}{1}%
  {\bf (\#\arabic{probno}\alph{partno})}\ \ \bf\boldmath #1\unboldmath
  \end{framed}}

\newcommand{\prob}{
  \vskip10bp%
  \setcounter{partno}{0}%
  \addtocounter{probno}{1}%
  {\bf Problem~\#{\arabic{probno}}}\quad}
\newcommand{\probpart}{%\rule{0in}{0in}\\ \phantom{xxx}
  \addtocounter{partno}{1}%
  {\bf (\#\arabic{probno}\alph{partno})}\ \ }
\newcommand{\probcont}{%
  {\bf Problem~\#{\arabic{probno}}}~(\emph{continued})}
\newcommand{\probo}{
  \setcounter{partno}{0}%
  \addtocounter{probno}{1}%
  {\bf (\#\arabic{probno})}\ \ }

\newcommand{\abs}[1] {
  \left| #1 \right|}
