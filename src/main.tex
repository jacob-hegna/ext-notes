%% LaTeX header file for Math 821, Algebraic Topology
%% JLM 1/9/18
%% Feel free to use or modify these macros as you see fit.

% General stuff
\documentclass[reqno]{amsart}
\usepackage{amssymb,amsmath,amsthm,graphics,hyperref,mathrsfs,stmaryrd,dsfont,youngtab,tcolorbox,framed,blkarray,color,fullpage}
\usepackage[all]{xy}
\usepackage{enumerate}
\usepackage{bbm}
\numberwithin{equation}{section}
\renewcommand{\phi}{\varphi}

%%%%%%%%%%%%%%%%    Macros for figures   %%%%%%%%%%%%%%%%

%% I like to keep all my figures in a subfolder called "figs".  If you keep figures in the same folder as the source .tex tile, remove "figs/" from the macros.

%% Usage: \includefigure{height}{width}{filename}
\newcommand{\includefigure}[3]{{
  \begin{center}
  \resizebox{#1}{#2}{\includegraphics{{figs/#3}}}
  \end{center}}}
\newcommand{\includefigurewithinmath}[3]{{
  \resizebox{#1}{#2}{\includegraphics{{figs/#3}}}}}

%%%%%%%%%%%%%%%%   Vertical spacing   %%%%%%%%%%%%%%%%
\newcommand{\hugepad}{\rule[-14mm]{0mm}{30mm}}
\newcommand{\bigpad}{\rule[-5mm]{0mm}{15mm}}
\newcommand{\bigpadup}{\rule{0mm}{10mm}}
\newcommand{\bigpaddown}{\rule[-5mm]{0mm}{5mm}}
\newcommand{\smallpad}{\rule[-1.5mm]{0mm}{5mm}}
\newcommand{\pad}{\rule[-3mm]{0mm}{8mm}}
\newcommand{\padup}{\rule{0mm}{5mm}}
\newcommand{\paddown}{\rule[-3mm]{0mm}{2mm}}
\newcommand{\blank}{\rule{1.25in}{0.25mm}}
\newcommand{\yell}[1]{\begin{framed}{#1}\end{framed}}
\newcommand{\bang}{$\bullet$\quad}
\newcommand{\indnt}{\phantom{.}\qquad}
\newcommand{\littleline}{\begin{center}\rule{4in}{0.5bp}\end{center}}
\newcommand{\defterm}[1]{\boldmath\textbf{#1}\unboldmath}

\newcommand{\icol}[1]{% inline column vector
  \left(\begin{smallmatrix}#1\end{smallmatrix}\right)%
}

%%%%%%%%%%%%%%%%   Colors for TikZ and text  %%%%%%%%%%%%%%%%%

\definecolor{light}{gray}{.75}
\definecolor{med}{gray}{.5}
\definecolor{dark}{gray}{.25}
\newcommand{\Red}[1]{{\color{red}{#1}}}
\newcommand{\RED}[1]{{\color{red}{\boldmath\textbf{#1}\unboldmath}}}
\newcommand{\Blue}[1]{{\color{blue}{#1}}}
\newcommand{\BLUE}[1]{{\color{blue}{\boldmath\textbf{#1}\unboldmath}}}

%%%%%%%%%%%%%%%%%%%%   Hyperlinks   %%%%%%%%%%%%%%%%%%%%

\newcommand{\hreftext}[2]{\href{#1}{\Blue{#2}}}
\newcommand{\hrefurl}[2]{\href{#1}{\Blue{\tt #2}}}

%%%%%%%%%%%%%%%%%%%%   Math operators   %%%%%%%%%%%%%%%%%%%%

\DeclareMathOperator{\Spec}{Spec}
\DeclareMathOperator{\interior}{Int}
\DeclareMathOperator{\Mod}{Mod}
\newcommand{\etale}{\'etal{e\xspace}}

\DeclareMathOperator{\Ab}{Ab}
\DeclareMathOperator{\coker}{coker}
\DeclareMathOperator{\Comm}{Comm}
\DeclareMathOperator{\conv}{conv}
\DeclareMathOperator{\diag}{diag}
\DeclareMathOperator{\Ext}{Ext}
\DeclareMathOperator{\Gr}{Gr}
\DeclareMathOperator{\Hom}{Hom}
\DeclareMathOperator{\im}{im}
\DeclareMathOperator{\Link}{link}
\DeclareMathOperator{\nullspace}{nullspace}
\DeclareMathOperator{\Poin}{Poin}
\DeclareMathOperator{\rank}{rank}
\DeclareMathOperator{\rel}{rel}
\DeclareMathOperator{\Tor}{Tor}
\DeclareMathOperator{\tr}{tr}
\DeclareMathOperator{\sd}{sd}
\DeclareMathOperator{\Star}{star}
\DeclareMathOperator{\support}{support}
\DeclareMathOperator{\vol}{vol}

%%%%%%%%%%%%%%%%%%%%   Theorem environments with automatic numbering

\newtheorem{theorem}{Theorem}[section]
\newtheorem{proposition}[theorem]{Proposition}
\newtheorem{lemma}[theorem]{Lemma}
\newtheorem{corollary}[theorem]{Corollary}
\newtheorem{criterion}[theorem]{Criterion}
\theoremstyle{definition}
\newtheorem{definition}[theorem]{Definition}
\newtheorem{example}[theorem]{Example}
\newtheorem{remark}[theorem]{Remark}
\newtheorem{problem}[theorem]{Problem}
\theoremstyle{remark}
\newtheorem{exercise}{Exercise}

%%%%%%%%%%%%%%%%%%%%   Unnumbered theorem proclamations

\newcommand{\cor}{{\bf Corollary: }}
\newcommand{\defn}{{\bf Definition: }}
\newcommand{\defnthm}{{\bf Definition/Theorem: }}
\newcommand{\defnprop}{{\bf Definition/Proposition: }}
\newcommand{\defns}{{\bf Definitions: }}
\newcommand{\exa}{{\bf Example: }}
\newcommand{\fact}{{\bf Fact: }}
\newcommand{\lem}{{\bf Lemma: }}
\newcommand{\notn}{{\bf Notation: }}
\newcommand{\obs}{{\bf Observation: }}
\newcommand{\note}{{\bf Note: }}
\newcommand{\prop}{{\bf Proposition: }}
\newcommand{\rmk}{{\bf Remark: }}
\newcommand{\skpr}{\emph{Sketch of proof: }}
\newcommand{\soln}{{\bf Solution: }}
\newcommand{\altsoln}{{\bf Alternate Solution: }}
\newcommand{\altaltsoln}{{\bf Another Alternate Solution: }}
\newcommand{\thm}{{\bf Theorem: }}

\newcommand{\half}{\frac{1}{2}}
\newcommand{\thalf}{\tfrac{1}{2}}
\newcommand{\dhalf}{\dfrac{1}{2}}

\newcommand{\0}{\emptyset}
\newcommand{\bd}{\partial} %% boundary
\newcommand{\dju}{\ensuremath{\mathaccent\cdot\cup}} %% disjoint union
\newcommand{\eqdef}{\overset{\rm def}{=}}
\newcommand{\ergo}{\therefore}  %% just because it's shorter
\newcommand{\excise}[1]{} % for commenting out large blocks of LaTeX source
\newcommand{\intersection}{\cap}
\newcommand{\isom}{\cong}  %% homeomorphism
\newcommand{\htop}{\simeq} %% homotopy equivalence
\newcommand{\ov}[1]{\overline{{#1}}}
\newcommand{\qandq}{\quad\text{and}\quad}
\newcommand{\qqandqq}{\qquad\text{and}\qquad}
\newcommand{\qand}{\quad\text{and}}
\newcommand{\ray}{\overrightarrow}
\newcommand{\rchi}{\raisebox{0.05cm}{\ensuremath{\chi}}}  %% because chi looks funny otherwise
\newcommand{\sm}{\setminus}
\newcommand{\st}{~|~}		%% ``such that''
\newcommand{\www}{\mathsf w}	%% winding number
\newcommand{\x}{\times}
\newcommand{\union}{\cup}
\newcommand{\zerovec}{\mathbf{0}}

%% Arrows
\newcommand{\into}{\hookrightarrow} % injection
\newcommand{\onto}{\twoheadrightarrow} % surjection
\newcommand{\inj}{\hookrightarrow} % injection
\newcommand{\surj}{\twoheadrightarrow} % surjection

%% Arrows for use with xy
\newcommand{\rto}{\ar[r]}
\newcommand{\rrto}{\ar[rr]}
\newcommand{\lto}{\ar[l]}
\newcommand{\llto}{\ar[ll]}
\newcommand{\uto}{\ar[u]}
\newcommand{\uuto}{\ar[uu]}
\newcommand{\dto}{\ar[d]}
\newcommand{\ddto}{\ar[dd]}
\newcommand{\dlto}{\ar[dl]}
\newcommand{\drto}{\ar[dr]}
\newcommand{\ulto}{\ar[ul]}
\newcommand{\urto}{\ar[ur]}
\newcommand{\ddlto}{\ar[ddl]}
\newcommand{\dllto}{\ar[dll]}
\newcommand{\ddrto}{\ar[ddr]}
\newcommand{\drrto}{\ar[drr]}
\newcommand{\uulto}{\ar[uul]}
\newcommand{\ullto}{\ar[ull]}
\newcommand{\uurto}{\ar[uur]}
\newcommand{\urrto}{\ar[urr]}
\newcommand{\ddllto}{\ar[ddll]}
\newcommand{\uullto}{\ar[uull]}

%% Straight lines for use with xy
\newcommand{\rlineto}{\ar@{-}[r]}
\newcommand{\rrlineto}{\ar@{-}[rr]}
\newcommand{\llineto}{\ar@{-}[l]}
\newcommand{\lllineto}{\ar@{-}[ll]}
\newcommand{\ulineto}{\ar@{-}[u]}
\newcommand{\uulineto}{\ar@{-}[uu]}
\newcommand{\dlineto}{\ar@{-}[d]}
\newcommand{\ddlineto}{\ar@{-}[dd]}
\newcommand{\dllineto}{\ar@{-}[dl]}
\newcommand{\drlineto}{\ar@{-}[dr]}
\newcommand{\ullineto}{\ar@{-}[ul]}
\newcommand{\urlineto}{\ar@{-}[ur]}
\newcommand{\ddllineto}{\ar@{-}[ddl]}
\newcommand{\dlllineto}{\ar@{-}[dll]}
\newcommand{\ddrlineto}{\ar@{-}[ddr]}
\newcommand{\drrlineto}{\ar@{-}[drr]}
\newcommand{\uullineto}{\ar@{-}[uul]}
\newcommand{\ulllineto}{\ar@{-}[ull]}
\newcommand{\uurlineto}{\ar@{-}[uur]}
\newcommand{\urrlineto}{\ar@{-}[urr]}

%% Boldface letters
\renewcommand{\aa}{\mathbf{a}}
\newcommand{\bb}{\mathbf{b}}
\newcommand{\pp}{\mathbf{p}}
\newcommand{\qq}{\mathbf{q}}
\newcommand{\vv}{\mathbf{v}}
\newcommand{\ww}{\mathbf{w}}
\newcommand{\xx}{\mathbf{x}}
\newcommand{\yy}{\mathbf{y}}
\newcommand{\zz}{\mathbf{z}}

%% Calligraphic letters
\newcommand{\A}{\mathcal{A}}
\newcommand{\B}{\mathcal{B}}
\newcommand{\C}{\mathcal{C}}
\newcommand{\M}{\mathcal{M}}
\renewcommand{\P}{\mathcal{P}}
\newcommand{\T}{\mathcal{T}}

%% Fancy script letters
\newcommand{\BB}{\mathscr{B}}
\newcommand{\CC}{\mathscr{C}}
\newcommand{\FF}{\mathscr{F}}
\newcommand{\II}{\mathscr{I}}
\newcommand{\LL}{\mathscr{L}}
\newcommand{\PP}{\mathscr{P}}
\renewcommand{\SS}{\mathscr{S}}
\newcommand{\TT}{\mathscr{T}}
\newcommand{\XX}{\mathscr{X}}

%% Blackboard bold letters
\newcommand{\Aa}{\mathbb{A}}
\newcommand{\Cc}{\mathbb{C}}
\newcommand{\Ff}{\mathbb{F}}
\newcommand{\Nn}{\mathbb{N}}
\newcommand{\Pp}{\mathbb{P}}
\newcommand{\Qq}{\mathbb{Q}}
\newcommand{\Rr}{\mathbb{R}}
\newcommand{\Ss}{\mathbb{S}}
\newcommand{\Zz}{\mathbb{Z}}
\newcommand{\IdMap}{\mathds{1}}

\newcommand{\RP}{\Rr P}
\newcommand{\CP}{\Cc P}
\newcommand{\RPN}{\Rr P^n}
\newcommand{\CPN}{\Cc P^n}
\newcommand{\RPn}{\Rr P^n}
\newcommand{\CPn}{\Cc P^n}

%% tilde H for reduced homology
\newcommand{\HH}{\tilde H}

%% big double angle brackets
\newcommand{\Langle}{\big\langle\!\!\big\langle}
\newcommand{\Rangle}{\big\rangle\!\!\big\rangle}

%% macros for automatic problem numbering
\newcounter{probno}
\setcounter{probno}{0}
\newcounter{partno}
\setcounter{partno}{0}

%% versions that don't print the number of points
\newcommand{\yellprob}[1]{%
  \bigskip\begin{framed}
  \setcounter{partno}{0}%
  \addtocounter{probno}{1}%
  {\bf Problem~\#{\arabic{probno}}}\quad\bf\boldmath #1\unboldmath
  \end{framed}}

\newcommand{\yellprobpart}[1]{
  \bigskip\begin{framed}%
  \addtocounter{partno}{1}%
  {\bf (\#\arabic{probno}\alph{partno})}\ \ \bf\boldmath #1\unboldmath
  \end{framed}}

\newcommand{\prob}{
  \vskip10bp%
  \setcounter{partno}{0}%
  \addtocounter{probno}{1}%
  {\bf Problem~\#{\arabic{probno}}}\quad}
\newcommand{\probpart}{%\rule{0in}{0in}\\ \phantom{xxx}
  \addtocounter{partno}{1}%
  {\bf (\#\arabic{probno}\alph{partno})}\ \ }
\newcommand{\probcont}{%
  {\bf Problem~\#{\arabic{probno}}}~(\emph{continued})}
\newcommand{\probo}{
  \setcounter{partno}{0}%
  \addtocounter{probno}{1}%
  {\bf (\#\arabic{probno})}\ \ }

\newcommand{\abs}[1] {
  \left| #1 \right|}


\usepackage[utf8]{inputenc}
\usepackage{mathtools}
\usepackage{multirow}
\usepackage{graphicx}
\usepackage{caption}
\usepackage{tikz-cd}

\tikzset{
    curarrow/.style={
    rounded corners=8pt,
    execute at begin to={every node/.style={fill=red}},
    to path={-- ([xshift=-50pt]\tikztostart.center)
    |- (#1)
    -| ([xshift=50pt]\tikztotarget.center)
    -- (\tikztotarget)}
    }
}

\graphicspath{ {src/images/} }

\begin{document}

\title{Notes on Ext}
\author{Jacob Hegna \\ \textbf{draft: \today}}
\date{\today}

\maketitle

These notes were used to deliver a lecture on derived functors and the universal
coefficient theorem to MATH $821$ at the University of Kansas. This is not
original work, and much of it is taken from two sources:
\begin{itemize}
    \item These notes by Peter May: http://www.math.uchicago.edu/~may/MISC/TorExt.pdf
    \item This mathoverflow post:
        https://mathoverflow.net/questions/1151/sheaf-cohomology-and-injective-resolutions.
\end{itemize}

\tableofcontents

\section{Conventions}

In these notes, a \textit{category} $\mathcal{A}$ is a collection of objects
$\mathop{Ob}(\mathcal{A})$ and, for any two objects $A, B \in \mathcal{A}$ a
collection of maps $\Hom_\mathcal{A}(A, B)$. It is usually at this point that
notes will mention that this definition often has some set-theoretic issues, and
say either
\begin{enumerate}
    \item we ignore all set theoretic issues
    \item all categories are assumed to be \textit{locally small}, which means
        all $\Hom$ objects are indeed sets, not proper classes
\end{enumerate}
these notes will take the first approach. The $\Hom$ sets describe formal arrows
from $A$ to $B$; we require that there be an identity arrow for every object,
and that there is an operation $\circ$ called \textit{composition} which is
associative. A \textit{functor} between categories is a map $\mathcal{F} :
\mathcal{A} \to \mathcal{B}$ which takes objects to objects and arrows to
arrows. In particular, the mapping of objects and arrows should play nicely, in
other words if $f : A \to B$ in $\mathcal{A}$, then $\mathcal{F}(f)$ should be a
map from $\mathcal{F}(A) \to \mathcal{F}(B)$. It should finally be a
homomorphism under composition.

We will often refer to the notion of an Abelian category. The definition is a
bit involved, and the reader is welcome to search it out on their own, but the
two most important things to remember about such categories are that
\begin{enumerate}
    \item They possess all kernels, cokernels, products, coproducts, and there
        exists a zero object.

    \item Any abelian category $\mathcal{A}$ has a fully faithful embedding into
        a subcategory of $\Mod(R)$. Note, however, that $R$ need not be
        commutative, so (for instance) things like tensor products need not make
        sense in every abelian category. In these notes, if a tensor product is
        used, the reader should assume that the category in question is
        (essentially) modules over a commutative ring. 
\end{enumerate}
These categories are a standard context for homological algebra.

\section{Derived functors}

\begin{definition}
    A \textit{cochain complex} in an abelian category $\mathcal{A}$ is a
    collection of objects $\{A_i\}_{i \in \Zz}$ where $A_i = 0$ for $i < 0$, and
    maps $\delta_i : A_i \to A_{i+1}$ which satisfy the \textit{cochain
        condition}, $\delta^2 = 0$.
\end{definition}

Often it is convient to consider the cochain complex as a $\Zz_+$-graded object
$A = \bigoplus A_i$ with the obvious map $\delta$. We say that the complex $(A,
\delta)$ is \textit{exact} if $\ker{\delta}=\im{\delta}$. It is \textit{short
    exact} if it degenerates to the complex $0 \to A \to B \to C \to 0$, and
\textit{long exact} otherwise.

Given a functor $\mathcal{F}$ and a short exact sequence $0 \to A \to B \to C
\to 0$, it is common to ask when the sequence
\begin{center}
    \begin{tikzcd}
        0 \ar[r] &
        \mathcal{F}(A) \ar[r] &
        \mathcal{F}(B) \ar[r] &
        \mathcal{F}(C) \ar[r] &
        0
    \end{tikzcd}
\end{center}
is exact. In general, it is more interesting to study those functors
$\mathcal{F}$ for which exactness fails. We say $\mathcal{F}$ is \textit{right
    exact} if the complex is exact everywhere but $\mathcal{F}(A)$, and
\textit{left exact} if it is exact everywhere but $\mathcal{F}(C)$.
$\mathcal{F}$ is \textit{exact} if it is both left and right exact.

\begin{example}
    An equivalence of categories is exact. The global section functor $\Gamma$
    is left exact. The functor which assigns to a $G$-module $M$ the group $M^G$
    is left exact. In general, if $\mathcal{F}$ is a right adjoint of some
    functor $\mathcal{G}$, then it is left exact and $\mathcal{G}$ is right
    exact.

\end{example}

\begin{remark}
    Actually, the previous statements are a consequence of a much deeper fact:
    left adjoints preserve \textit{all} colimits, and right adjoints preserve
    \textit{all} limits. This is one of the reasons that in any category
    enriched over $\mathbf{Set}$, products look just like they do
    set-theoretically. Recall that functors $\mathcal{F} : \mathcal{A} \to
    \mathcal{B}$ and $\mathcal{G} : \mathcal{B} \to \mathcal{A}$ are said to be
    \textit{adjoint} if
    \[
        \Hom_\mathcal{B}(\mathcal{F}(A), B) \simeq \Hom_\mathcal{A}(A,
        \mathcal{G}(B))
    \]
    For $A \in \mathcal{A}$ and $B \in \mathcal{B}$.
\end{remark}

Studying the failure of a short exact sequence to be exact after an application
of $\mathcal{F}$ has given rise to some particularly interesting mathematics.
For example, in algebraic geometry, a particular functor $\Gamma$ called the
``global sections'' functor is left exact, and studying why it fails to be right
exact gives detailed information about the geometric object underlying the
functor. Before introducing derived functors, however, we require the following
definition.

\begin{definition}
    An object $A$ in an abelian category is \textit{injective} if, for all $f :
    G \to A$ and monomorphisms $h : G \to G'$, there is an $f'$ for which
    \begin{center}
        \begin{tikzcd}
            & A \\
            G' \ar[ur, "\exists f'", dashed] &
            G \ar[u, "f"] \ar[l, "h"] &
            0 \ar[l]
        \end{tikzcd}
        \quad commutes.
    \end{center}
    In this context, we say that $f'$ is a lift of $h$ through $f$.
\end{definition}

An abelian category has \textit{enough injectives} if, for any object $A$, there
is a monomorphism $i : A \to I$ where $I$ is injective. A \textit{monomorphism}
is a map which behaves like an injection. More specifically, $i : X \to Y$ is a
monomorphism if and only if commutivatiy of
\begin{center}
    \begin{tikzcd}
        A \ar[r, shift left, "\alpha"] \ar[r, swap, shift right, "\beta"] &
        X \ar[r, "i"] &
        Y
    \end{tikzcd}
\end{center}
implies that $\alpha = \beta$. Equivalently, one could say that $\ker{i} = 0$
(whatever this means in an arbitrary abelian category).

\begin{remark}
    Consider the exact sequence $0 \to A \to B \to C \to 0$. Suppose $A$ is
    injective. In the diagram used to define injectivity, take $G=A$, $f$ the
    identity on $A$, and $h$ the map from $A \to B$. Injectivity of $A$ is
    precisely the statement that the short exact sequence splits. If
    $\mathcal{F}$ is a covariant functor, then splitting of the sequence implies
    the identity $\mathcal{F}(C) \to \mathcal{F}(C)$ factors through
    $\mathcal{F}(B) \to \mathcal{F}(C)$. If $\mathcal{F}$ is left-exact, then
    the sequence $0 \to \mathcal{F}(A) \to \mathcal{F}(B) \to \mathcal{F}(C) \to
    0$ is exact.
\end{remark}

Now, suppose we wish to develop a (co)homology theory which captures the failure
of $\mathcal{F}$ to preserve (right) exactness. We would wish that
$H^0(X)=\mathcal{F}(X)$, and we would also want that short exact sequences give
rise to long exact sequences in cohomology. More concretely, if $0 \to A \to B
\to C \to 0$ is exact, then we would want exactness of
\begin{center}
    \begin{tikzcd}
        0 \ar[r] &
        H^0(A) \ar[r] &
        H^0(B) \ar[r] &
        H^0(C) \ar[r] &
        H^1(A) \ar[r] &
        H^1(B) \ar[r] &
        H^1(C) \ar[r] &
        \cdots
    \end{tikzcd}
\end{center}
Finally, if $0 \to \mathcal{F}(A) \to \mathcal{F}(B) \to \mathcal{F}(C) \to 0$
is exact, then all the higher cohomology objects of $A$ should vanish. In
general, if $\mathcal{F}$ is left exact, we desire a cohomology theory. If it is
right exact, we desire a homology theory. For the purposes of these notes, we
will primarily be concerned with the cohomological case.

How could we calculate the value of these objects $H^i({-})$? By fiat, we know
that $H^0(X) = \mathcal{F}(X)$. To calculate $H^1(X)$, let us assume first we
have a theory which satisfies the long exact sequence of cohomology. If we embed
$X$ into an injective object $I^0$, we have a short exact sequence
\begin{center}
    \begin{tikzcd}
        0 \ar[r] &
        X \ar[r] &
        I^0 \ar[r] &
        K^1 \ar[r] &
        0
    \end{tikzcd}
\end{center}
Taking the long exact sequence, we have that the higher $H^i(I^0)$ terms all
vanish (by Remark 1.5), so the sequence degenerates to isomorphisms $H^i(K^1)
\cong H^{i-1}(X)$ for $i > 1$. In the $i=1$ case, we have a sequence
\begin{center}
    \begin{tikzcd}
        0 \ar[r] &
        \mathcal{F}(X) \ar[r] &
        \mathcal{F}(I^0) \ar[r] &
        \mathcal{F}(K^1) \ar[r] &
        H^1(X) \ar[r] &
        0
    \end{tikzcd}
\end{center}
This gives us that $H^1(X) = \mathcal{F}(K^1) / \im(\mathcal{F}(I^0))$, which
lets us compute $H^1(X)$ in terms of $\mathcal{F}$ itself. But how do we
calculate the higher cohomology of $K^1$ needed to recover the higher cohomology
groups of $X$? We should embed $K^1$ into an injective object $I^1$ (with
cokernel $K^2$), generating isomorphisms $H^1(K^1) \cong \mathcal{F}(K^2) /
\im(\mathcal{F}(I^1))$. Repeating this process, we generate an \textit{injective
    resolution} of the object $X$
\begin{center}
    \begin{tikzcd}
        & & & K^1 \ar[dr]

        \\

        0 \ar[r] &
        X \ar[r] &
        I^0 \ar[rr, dashed] \ar[ur] &
        & I^1 \ar[rr, dashed] \ar[dr] &
        & I^2 \ar[r] &
        \cdots

        \\

        & & & & & K^2 \ar[ur]
    \end{tikzcd}
\end{center}
The resolution itself is the chain complex $0 \to X \to I^\bullet$. Note that,
essentially by construction, the resolution is exact. Now, calculating the
higher cohomology groups is reduced to induction. By definition,
\[
    H^i(X) = H^{i-1}(K^1) = H^{i-2}(K^2) = \cdots = H^1(K^{i-1}) =
    \mathcal{F}(K^i) / \im(\mathcal{F}(K^{i-1}))
\]
We can simplify this further to remove reference to the $K^i$ objects. As
$\mathcal{F}$ is left-exact, the sequence $0 \to K^i \to I^i \to I^{i+1}$ gives
rise to an isomorphism $\mathcal{F}(K^i) \cong \ker(\mathcal{F}(I^i))$. Thus,
after all this work, we can compute $H^i(X)$ by taking an injective resolution
$0 \to X \to I^\bullet$ and computing the cohomology of the complex $0 \to
I^\bullet$.

There are many things to check. Primarily, one should ask the following
question: if one takes two distinct resolutions of $X$, does the computation of
$H^i(X)$ differ? The answer is given by the following lemma, when one takes $f :
X \to X$ to be the identity.

\begin{lemma}
    Let $\alpha : X \to I^\bullet$ and $\beta : Y \to J^\bullet$ be injective
    resolutions of $X$ and $Y$, respectively. A map $f : X \to Y$ induces a map
    $\tilde{f} : I^\bullet \to J^\bullet$ which satisfies $\tilde{f} \circ
    \alpha = \beta \circ f$, and this map is unique up to chain homotopy.
\end{lemma}

The lemma and the previous discussion motivate the following definition. Before
we make it, fix once and for all an injective resolution for each object of our
ambient category.

\begin{definition}
    Let $\mathcal{F}$ be a left exact functor of abelian categories, where the
    domain has enough injectives. The \textit{right derived functors} of
    $\mathcal{F}$ are the objects formed by taking an injective resolution $0
    \to X \to I^\bullet$, chopping off $X$, applying $\mathcal{F}$ to $0 \to
    I^\bullet$, and taking cohomology. Symbolically,
    \[
        R^i\mathcal{F}(X) = h^i(I^\bullet) = \frac{\ker\left( \mathcal{F}(I^i
                \to I^{i+1})\right)}{\im\left( \mathcal{F}(I^{i-1} \to
                I^i)\right)} \]
    where $0 \to X \to I^\bullet$ is the chosen injective resolution of $X$.
\end{definition}

The functors $R^i \mathcal{F}({-})$ depend on the choice of resolution for each
object. But, by Lemma 1.5, different choices yield isomorphic functors.

\begin{remark}
    In the previous work, we (crucially) used the fact if $I$ was injective,
    then it has no higher cohomology. However, the same game would have worked
    if we had taken an $\mathcal{F}$-\textit{acyclic} resolution $J^\bullet$ of
    $X$, where $R^{i}\mathcal{F}(J^k) = 0$ for $i, k > 0$. This is particularly
    useful in the case of sheaf cohomology, where often we compute the
    cohomology groups $H^i(X, \mathcal{F})$ by taking flasque or fine
    resolutions.
\end{remark}

Although we have motivated this definition, we still need to formally prove some
results about it.

\begin{theorem}
    The functors $R^i \mathcal{F}({-})$ satisfy the following properties
    \begin{enumerate}
        \item $R^0 \mathcal{F}(X) = \mathcal{F}(X)$.
        \item Short exact sequences give rise to long exact sequences under $R^i
            \mathcal{F}$.
    \end{enumerate}

    \begin{proof}
        The kernel of the map $\mathcal{F}(I^0 \to I^1)$ is the object
        $\mathcal{F}(X)$. The image of $\mathcal{F}(I^{-1} \to I^0)$, taking
        $I^{-1} = 0$, is $0$. This proves (1).

        Given a short exact sequence $0 \to A \to B \to C \to 0$, we take
        injective resolutions of each object, and the Lemma induces maps between
        the resolutions. Diagramatically, we have
        \begin{center}
            \begin{tikzcd}
                0 \ar[r] & A \ar[r] \ar[d] & B \ar[r] \ar[d] & C \ar[r] \ar[d] & 0 \\
                0 \ar[r] & I_A^0 \ar[r] \ar[d] & I_B^0 \ar[r] \ar[d] & I_C^0 \ar[r] \ar[d] & 0 \\
                0 \ar[r] & I_A^1 \ar[r] \ar[d] & I_B^1 \ar[r] \ar[d] & I_C^1 \ar[r] \ar[d] & 0 \\
                & \vdots & \vdots & \vdots
            \end{tikzcd}
        \end{center}
        Apply the snake lemma to each row. This gives the desired long exact
        sequence. Naturality of the snake lemma yields naturality of the induced
        long exact sequence.
    \end{proof}
\end{theorem}

\begin{example}
    The core example of these notes is the $\Ext$ functor. However, note that
    this definition is central to many aspects of commutative algebra.
    Cohomology of groups and of sheaves (an algebraic tool to keep track of
    local data on a topological space) are described explicitly by derived
    functors. 
\end{example}

\section{Ext}

Let $R$ be a commutative ring with unity. The category $\Mod(R)$ of $R$-modules
is abelian. If $0 \to A \to B \to C \to 0$ is an exact sequence of $R$-modules,
the the sequences
\begin{center}
    \begin{tikzcd}
        0 \ar[r] &
        \Hom_R(M, A) \ar[r] &
        \Hom_R(M, B) \ar[r] &
        \Hom_R(M, C)
    \end{tikzcd}

    \begin{tikzcd}
        0 \ar[r] &
        \Hom_R(C, N) \ar[r] &
        \Hom_R(B, N) \ar[r] &
        \Hom_R(A, N)
    \end{tikzcd}
\end{center}
are exact for $R$-modules $M, N$. In other words, the bifunctor $\Hom_{R}({-}, {-})$ is left exact in both arguments.

\begin{definition}
    The functors $\Ext_R^i(A, {-})$ (resp. $\Ext_R^i({-}, B)$) are defined to be the right derived functors of $\Hom_R(A, {-})}$ (resp. $\Hom_R({-}, B)}$).
\end{definition}

Note that, in the case of $\Hom_R({-}, B)$, taking the right derived functors amounts to taking a projective resolution instead of an injective resolution.

Concretely, the modules $\Ext_R^i(M, N)$ classify extensions of the module $N$ by $M$ of length $i$. So, $\Ext_R^i(M, N)$ is in bijection with the set of $\{E_j\}$'s for which
\begin{center}
    \begin{tikzcd}
        0 \ar[r] &
        N \ar[r] &
        E_1 \ar[r] &
        \cdots \ar[r] &
        E_j \ar[r] &
        \cdots \ar[r] &
        E_i \ar[r] &
        M \ar[r] &
        0
    \end{tikzcd}
\end{center}

Given extensions $\{E_i\} \in \Ext_R^n(N, P)$ and $\{R_j\} \in \Ext_R^m(M, N)$, one can form an extension of length $m + n$ in $\Ext_R^{m+n}(M, P)$ by
\begin{center}
    \begin{tikzcd}
        &&&&& N \ar[dr] \\

        0 \ar[r] &
        P \ar[r] &
        E_1 \ar[r] &
        \cdots \ar[r] &
        E_n \ar[ur] \ar[rr, dashed] &
        &R_1 \ar[r] &
        \cdots \ar[r] &
        R_m \ar[r] &
        M \ar[r] &
        0
    \end{tikzcd}
\end{center}
This (informally) describes a pairing
\[
    \Ext_R^n(N, P) \otimes \Ext_R^m(M, N) \longrightarrow \Ext_R^{m+n}(M, P)
\]
called the \textit{Yoneda product}. Observe that if we take $N=M=P$, then we obtain a multiplicative structure on the graded module $\Ext_R^{\ast}(M, M)$. This is somewhat analogous to the cup product in in singular cohomology. We may return to this later.

\begin{remark}
    Let $M, N, I, P$ be $R$-modules where $I$ is injective and $P$ is projective. The following properties follow from general derived functor nonsense.

    \begin{enumerate}
        \item $\Ext_R^0(M, N) = \Hom_R(M, N)$.
        \item $\Ext_R^i(P, N) = 0$ for $i > 0$.
        \item $\Ext_R^i(M, I) = 0$ for $i > 0$.
    \end{enumerate}
\end{remark}

We now begin the setup for the universal coefficient theorem. Note that, induced by the obvious short exact sequences, we have that the following are exact
\begin{equation}
    \begin{tikzcd}
        0 \ar[r] &
        \Hom(B_{n-1}, G) \ar[r] &
        \Hom(C_n, G) \ar[r] &
        \Hom(Z_n, G) \ar[r] &
        0
    \end{tikzcd}
\end{equation}

\begin{equation}
    \begin{tikzcd}
        0 \ar[r] &
        \Hom(H_n, G) \ar[r] &
        \Hom(Z_n, G) \ar[r] &
        \Hom(B_n, G) \ar[r] &
        \Ext^1(H_n, G) \ar[r] &
        0
    \end{tikzcd}
\end{equation}

\begin{theorem}[Universal Coefficients]
    For a PID $R$, the sequence
    \begin{center}
        \begin{tikzcd}
            0 \ar[r] &
            \Ext_R^1(H_{n-1}, G) \ar[r, "j"] &
            H^n(\Hom_R(C_\ast, G)) \ar[r, "h"] &
            \Hom_R(H_n, G) \ar[r] &
            0
        \end{tikzcd}
    \end{center}
    of $R$-modules is exact. It splits, although not naturally with respect to the grading of $\delta$.
\end{theorem}

In the proof, we have omitted the subscripts on $\Hom$ and $\Ext$ for clarity. Note that the general case of universal coefficients is a spectral sequence, but if $R$ is a PID, then it degenerates to the above short exact sequence.

\begin{proof}
    Consider the following commutative diagram.\footnote{Proof taken from http://ckottke.ncf.edu/docs/exttoruct.pdf} The middle row is the sequence (2.1), the first column is the sequence (2.2) in degree $n-1$, the third colum is the sequence (2.2) in degree $n$. The top and bottom rows are each sequence (2.1), in degrees $n$ and $n-1$, respectively.
    \begin{center}
        \begin{tikzcd}
            & 0 \\

            & \Ext^1(H_{n-1}, G) \ar[u] \ar[dr, "j", dotted] & \Hom(C_{n+1}, G) & \Hom(B_n, G) \ar[l, "\delta"] & 0 \ar[l] \\

            0 \ar[r] & \Hom(B_{n-1}, G) \ar[r, "\delta"] \ar[u] & \Hom(C_n, G) \ar[dr, "h", dotted] \ar[u, "\delta"] \ar[r, "i^\ast"] & \Hom(Z_n, G) \ar[l, bend right, dashed] \ar[u, "i^\ast"] \ar[r] & 0 \\

            0 & \Hom(Z_{n-1}, G) \ar[l] \ar[u, "i^\ast"] & \Hom(C_{n-1}, G) \ar[u, "\delta"] \ar[l, "i^\ast"] & \Hom(H_n, G) \ar[u, "q^\ast"] \\

            & & & 0 \ar[u]
        \end{tikzcd}
    \end{center}

    The map $j$ is defined by taking an element $x \in \Ext^1(H_{n-1}, G)$, lifting it to $x_0 \in \Hom(B_{n-1}, G)$, and setting $j(x) = \delta(x_0)$. It is well-defined on cohomology classes because if we had lifted $x$ to $x_1$, then $x_0 - x_1$ is in the image of the map from $\Hom(Z_{n-1}, G)$ and thus is $0$ in cohomology.

    The map $h$ is defined by taking an element $y \in \Hom(C_n, G)$ then taking the preimage of $i^\ast(y)$ under $q^\ast$. It exists, because $i^\ast(y)$ is in the kernel of $i^\ast$ by a chasing the commutative square.

    The composition $h \circ j = 0$ because it takes the two arrows in the middle exact sequence. $\ker{h} = \im{j}$ because $h(y) = 0$ implies $i^\ast(y)=0$ so we can lift $y$ to $\Hom(B_{n-1}, G)$ and project it onto $\Ext^1(H_{n-1}, G)$. We leave the reader to ponder exactness at the endpoints of the sequence.

    The splitting of the sequence is due to the splitting of the middle row. It is not natural because of the failure of naturality of the middle row (it splits between gradings of the chain complex).
\end{proof}

Remarkably, none of the proof of Universal Coefficients relies on the properties of $\Hom$ or $\Ext$ outside of general properties of derived functors. Thus, we can rewrite the statement of the theorem to be:

\begin{theorem}
    Let $R$ be a PID, and $C_\bullet$ a chain complex in $\Mod(R)$. If $\mathcal{F}, \mathcal{G} : \Mod(R) \to \mathcal{A}$ are functors such that $\mathcal{F}$ is contravariant and left exact, and $\mathcal{G}$ is covariant and right exact, then
    \begin{center}
        \begin{tikzcd}
            0 \ar[r] &
            R^1\mathcal{F}(H_{n-1}(C_\bullet)) \ar[r] &
            H^n(\mathcal{F}(C_\bullet)) \ar[r] &
            \mathcal{F}(H_n(C_\bullet)) \ar[r] &
            0
            \\
            0 \ar[r] &
            \mathcal{G}(H_n(C_\bullet)) \ar[r] &
            H_n(\mathcal{G}(C_\bullet)) \ar[r] &
            L^1\mathcal{G}(H_{n-1}(C_\bullet)) \ar[r] &
            0
        \end{tikzcd}
    \end{center}
    are exact. In particular, if the homology groups of $C_\bullet$ are free, then $\mathcal{G}$ and $H_n$ commute over $C_\bullet$.
\end{theorem}

\iffalse
\begin{theorem}
Suppose $p$ is a prime which satisfies $p \equiv 4 \pmod{4}$. Then, there is no integer $n$ for which $p$ divides $n^2 + 1$.

    \begin{proof}
        Note that $(p-1)/2$ is odd, as by assumption, $(p-1)/2 \equiv 1 \pmod{4}$. Assume that that there existed such an $n$ where $p$ divides $n^2 + 1$. By basic modular arithmetic, $n^2 \equiv -1 \pmod{p}$. So,
        \[
            (n^2)^{(p-1)/2} = (n^2)^k \equiv -1 \pmod{p}
        \]
        as $k$ odd. However, by Fermat's little theorem, $(n^2)^{(p-1)/2} = n^{p-1} \equiv 1 \pmod{p}$. Now, $1 \equiv -1 \pmod{p}$ is impossible as $p \equiv 3 \pmod{4}$. So, such $n$ cannot exist.
    \end{proof}
\end{theorem}
\fi

\end{document}
